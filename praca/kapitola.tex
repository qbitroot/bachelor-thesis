\chapter{Jadro a členenie práce}

\label{kap:clenenie} % id kapitoly pre prikaz ref

V tejto kapitole si povieme niečo o jadre práce a o jej členení. Pod
jadrom práce rozumieme text medzi kapitolami Úvod a Záver.  Jadro
práce má byť členené na kapitoly, podkapitoly a podobne a tieto majú
byť číslované.  V zdrojovom kóde v súbore \verb'kapitola.tex' nájdete
ukážky použitých príkazov LaTeXu potrebných na písanie nadpisov, 
podnadpisov a nečíslovaných zoznamov.

\section{Jadro práce podľa smernice}
\label{sec:jadro} % id podkapitoly, ak bz sme sa na nu chceli neskor odkazovat

Smernica o záverečných prácach \cite[článok 5]{smernica} popisuje typické členenie jadra práce na nasledovné časti:
\begin{itemize}
\item  súčasný stav riešenej problematiky doma a v zahraničí,
\item  cieľ práce,
\item  metodika práce a metódy skúmania,
\item  výsledky práce, 
\item  diskusia. 
\end{itemize}

Hoci v niektorých študijných odboroch je vyžadované členenie práce na kapitoly uvedené v smernici, na informatických programoch nie je nutné toto členenie dodržiavať a môžete text rozdeliť do kapitol podľa potrieb konkrétnej témy. Napríklad ak robíte niekoľko analýz, pre každú môžete mať jednu kapitolu, v ktorej popíšete metodiku aj výsledky. Celkovo však práca musí zahŕňať texty týkajúce sa všetkých týchto odporúčaných častí. Pre popis súčasného stavu problematiky smernica odporúča rozsah asi 30\% práce, pre výsledky a diskusiu asi 30--40\%. Pokiaľ je vaša diskusia k výsledkom kratšia, môžete ju zahrnúť do záveru.

\section{Ďalšie členenie}

Tu uvádzame príklad príkazu na pridanie podnadpisu. Ide len o ukážku
použitia príslušných príkazov v LaTeXu. Vo vašej práci by ste mali
spravidla nemali mať podkapitoly s textom iba na pár riadkov a mali by
ste sa tiež vyvarovať prílišnému používaniu odrážkových alebo
číslovaných zoznamov.

\subsection{Ukážka podnadpisu}
\label{subsec:podnadpis}

Tu by pokračoval text podkapitoly \ref{subsec:podnadpis}.
