\documentclass[12pt, twoside]{book}
%\documentclass[12pt, oneside]{book}  % jednostranna tlac

%spravne nastavenie okrajov
\usepackage[a4paper,top=2.5cm,bottom=2.5cm,left=3.5cm,right=2cm]{geometry}
%zapnutie fontov pre UTF8 kodovanie
\usepackage[utf8]{inputenc}
\usepackage[T1]{fontenc}

%zapnutie slovenskeho delenia slov
%a automatickych nadpisov ako Obsah, Obrázok a pod. v slovencine
\usepackage[slovak]{babel} % vypnite pre prace v anglictine!

%nastavenie riadkovania podla smernice
\linespread{1.25} % hodnota 1.25 by mala zodpovedat 1.5 riadkovaniu

% balicek na vkladanie zdrojoveho kodu
\usepackage{listings}
% ukazky kodu su cislovane ako Listing 1,2,...
% tu je Listing zmenene na Algoritmus 1,2,...
\renewcommand{\lstlistingname}{Algoritmus}
% nastavenia balicka listings
% mozete pridat aj language=...
% na nastavenie najcastejsie pouzivaneho prog. jazyka
% takisto sa da zapnut cislovanie riadkov
\lstset{frame=lines}

% balicek na vkladanie obrazkov
\usepackage{graphicx}
% balicek na vkladanie celych pdf dokumentov, tu zadanie
\usepackage{pdfpages}
% balicek na spravne formatovanie URL
\usepackage{url}
% balicek na hyperlinky v ramci dokumentu
% zrusime farebne ramiky okolo liniek aby pdf
% vyzeralo rovnako ako tlacena verzia
\usepackage[hidelinks,breaklinks]{hyperref}



% -------------------
% --- Definicia zakladnych pojmov
% --- Vyplnte podla vasho zadania, rok ma byt rok odovzdania
% -------------------
\def\mfrok{2025}
\def\mfnazov{Fuzzing protokolu WebTransport}
\def\mftyp{Bakalárska práca}
\def\mfautor{Vladyslav Havriuk}
\def\mfskolitel{doc. RNDr. Martin Stanek, PhD.}

%ak mate konzultanta, odkomentujte aj jeho meno na titulnom liste
%\def\mfkonzultant{tit. Meno Priezvisko, tit. }  

\def\mfmiesto{Bratislava, \mfrok}

% študenti BIN a DAV odkomentujú príslušnú dvojicu riadkov
\def\mfodbor{ Informatika }
\def\program{ Informatika }
% pre BIN:
%\def\mfodbor{ Informatika a Biológia }
%\def\program{ Bioinformatika }
% pre DAV:
%\def\mfodbor{ Informatika a Matematika } 
%\def\program{ Dátová veda }

% Ak je školiteľ z FMFI, uvádzate katedru školiteľa, zrejme by mala byť aj na zadaní z AIS2
% Ak máte externého školiteľa, uvádzajte Katedru informatiky 
\def\mfpracovisko{ Katedra informatiky }

\begin{document}     
\frontmatter
\pagestyle{empty}

% -------------------
% --- Obalka ------
% -------------------

\begin{center}
\sc\large
Univerzita Komenského v Bratislave\\
Fakulta matematiky, fyziky a informatiky

\vfill

{\LARGE\mfnazov}\\
\mftyp
\end{center}

\vfill

{\sc\large 
\noindent \mfrok\\
\mfautor
}

\cleardoublepage
% --- koniec obalky ----

% -------------------
% --- Titulný list
% -------------------


\noindent

\begin{center}
\sc  
\large
Univerzita Komenského v Bratislave\\
Fakulta matematiky, fyziky a informatiky

\vfill

{\LARGE\mfnazov}\\
\mftyp
\end{center}

\vfill

\noindent
\begin{tabular}{ll}
Študijný program: & \program \\
Študijný odbor: & \mfodbor \\
Školiace pracovisko: & \mfpracovisko \\
Školiteľ: & \mfskolitel \\
% Konzultant: & \mfkonzultant \\
\end{tabular}

\vfill


\noindent \mfmiesto\\
\mfautor

\cleardoublepage
% --- Koniec titulnej strany


% -------------------
% --- Zadanie z AIS
% -------------------
% v tlačenej verzii s podpismi zainteresovaných osôb.
% v elektronickej verzii sa zverejňuje zadanie bez podpisov
% v pracach v angličtine anglické aj slovenské zadanie

\newpage 
\setcounter{page}{3}
\includepdf{images/zadanie.pdf}

% --- Koniec zadania


% -------------------
%   Čestné vyhlásenie a nepovinné poďakovanie
% -------------------
\newpage 
\pagestyle{plain}
\paragraph{Čestné vyhlásenie:}
Čestne vyhlasujem, že celú bakalársku prácu na tému \uv{\mfnazov},
vrátane všetkých jej príloh a obrázkov, som vypracoval
samostatne, a to s použitím literatúry uvedenej v priloženom zozname.

Pri príprave tejto práce boli tiež použité nástroje umelej
inteligencie za účelom asistovaného písania a úpravy textu. Nástroje umelej
inteligencie som použil v súlade s príslušnými právnymi
predpismi, akademickými právami a slobodami, etickými a morálnymi
zásadami za súčasného dodržania akademickej integrity. Som si
vedomý, že plne zodpovedám za správnosť výsledného textu.

\paragraph{Poďakovanie:} Ďakujem svojmu školiteľovi doc. RNDr. Martinovi Stankovi, PhD. za odborné vedenie, cenné rady a pripomienky pri vypracovaní tejto práce.

% --- Koniec poďakovania

% -------------------
%   Abstrakt - Slovensky
% -------------------
\newpage 
\section*{Abstrakt}

Fuzzing protokolov je technika používaná na testovanie implementácií protokolov. Bola úspešne použitá na identifikáciu implementačných chýb a bezpečnostných zraniteľností v rôznych protokoloch. Cieľom tejto práce je preskúmať metódy fuzzingu protokolov a aplikovať ich na reálne implementácie protokolu WebTransport. Výsledky sú analyzované a diskutované.

Táto práca sa zaoberá fuzzingom protokolov na identifikáciu chýb v implementáciách a zraniteľností. Predstavujeme vyvinutý fuzzer založený na BooFuzz pre protokol WebTransport, ktorý zahŕňa mutácie štruktúry správ a duplikáciu sekvencií. Príspevok spočíva v integrácii s aioquic pre QUIC podporu, testovaní na echo serveroch a analýze výsledkov, vrátane objavenia chyby v Rust knižnici. Výsledky ukazujú efektivitu prístupu v odhaľovaní paník v pracovných vláknach, prispievajúc k spoľahlivosti WebTransportu.

\paragraph*{Kľúčové slová:} fuzzing, WebTransport, QUIC, HTTP/3, testovanie protokolov, bezpečnosť
% --- Koniec Abstrakt - Slovensky


% -------------------
% --- Abstrakt - Anglicky 
% -------------------
\newpage 
\section*{Abstract}

Protocol fuzzing is a technique used to test protocol implementations. It has been successfully employed to identify implementation errors and security vulnerabilities in various protocols. The goal of this thesis is to explore methods of protocol fuzzing and apply them to real-world implementations of the WebTransport protocol. The results are analyzed and discussed.

This thesis addresses protocol fuzzing for identifying bugs in implementations and vulnerabilities. We present a developed fuzzer based on BooFuzz for the WebTransport protocol, which includes message structure mutations and sequence duplication. The contribution lies in the integration with aioquic for QUIC support, testing on echo servers, and analysis of results, including the discovery of a bug in a Rust library. The results demonstrate the effectiveness of the approach in detecting worker thread panics, contributing to the reliability of WebTransport.

\paragraph*{Keywords:} fuzzing, WebTransport, QUIC, HTTP/3, protocol testing, security

% --- Koniec Abstrakt - Anglicky

% -------------------
% --- Predhovor - v informatike sa zvacsa nepouziva
% -------------------
%\newpage 
%
%\chapter*{Predhovor}
%
%Predhovor je všeobecná informácia o práci, obsahuje hlavnú charakteristiku práce 
%a okolnosti jej vzniku. Autor zdôvodní výber témy, stručne informuje o cieľoch 
%a význame práce, spomenie domáci a zahraničný kontext, komu je práca určená, 
%použité metódy, stav poznania; autor stručne charakterizuje svoj prístup a svoje 
%hľadisko. 
%
% --- Koniec Predhovor


% -------------------
% --- Obsah
% -------------------


\cleardoublepage
\tableofcontents

% ---  Koniec Obsahu

% -------------------
% --- Zoznamy tabuliek, obrázkov, skratiek, slovník - nepovinné
% -------------------

\newpage 

\listoffigures
\listoftables

% ---  Koniec Zoznamov

\mainmatter
\pagestyle{headings}


\input uvod.tex 

\input suvisiace_prace.tex

\input metodologia.tex

\input vysledky.tex

\input zaver.tex

% -------------------
% --- Bibliografia
% -------------------


\newpage	

\backmatter

\thispagestyle{empty}
\clearpage
\addcontentsline{toc}{chapter}{Literatúra}
\bibliographystyle{plain}
\bibliography{literatura} 

%---koniec Referencii

% -------------------
%--- Prilohy---
% -------------------

%Nepovinná časť prílohy obsahuje materiály, ktoré neboli zaradené priamo  do textu. Každá príloha sa začína na novej strane.
%Zoznam príloh je súčasťou obsahu.
%
\input appendixA.tex

\end{document}
