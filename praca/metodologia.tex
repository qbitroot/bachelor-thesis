\chapter{Prehľad, metodológia a implementácia}
\label{kap:metodologia}

V tejto kapitole popisujeme protokol WebTransport, zvolenú metodológiu fuzzingu a detaily našej implementácie.

\section{Protokol WebTransport}

WebTransport funguje na základe QUIC protokolu, ktorý zabezpečuje spoľahlivý transport, multiplexovanie a šifrovanie, zatiaľ čo HTTP/3 pridáva vrstvu pre inicializáciu a riadenie kapsúl \cite{webtransport_http3}. Protokol zahŕňa stavy ako nadviazanie spojenia cez handshake, vytváranie bidirekcionálnych streamov pre dátový prenos, odosielanie datagramov pre nízko-latentné správy a manipuláciu chýb ako RESET\_STREAM alebo CONNECTION\_CLOSE.

Architektúra WebTransportu pozostáva z nasledujúcich vrstiev:
\begin{itemize}
    \item Aplikácia (WebTransport API) \cite{webtransport_api}
    \item HTTP/3 framing
    \item QUIC transport
    \item UDP sockety
\end{itemize}

Dáta prúdia cez streamy a datagramy medzi týmito vrstvami. Ako vznikajúci protokol je WebTransport stále v aktívnom vývoji, ale už prijatý v hlavných prehliadačoch, čo zdôrazňuje potrebu testovania na zabezpečenie spoľahlivosti.

\section{Metodológia fuzzingu}

Zvolený black/gray-box fuzzingový prístup nevyužíva inštrumentáciu, pretože fuzzujeme jazykovo-agnosticky, ale využíva znalosť fungovania WebTransportu na generovanie relevantných mutácií. Zameriava sa na:
\begin{itemize}
    \item Mutácie štruktúry správ
    \item Duplikáciu sekvencií
    \item Testovanie proti echo serverom
\end{itemize}

Detekcia chýb sa realizuje cez logy a pády bez prístupu k interným stavom servera.

\section{Implementácia}

Implementácia zahŕňa integráciu BooFuzz \cite{boofuzz_docs, boofuzz_github} s aioquic pre podporu QUIC, umožňujúc asynchrónnu komunikáciu a manipuláciu šifrovaných paketov. Upravili sme BooFuzz na podporu nepodporovaných funkcií ako duplikácia správ, pridajúc vlastné mutátory pre sekvencie kapsúl a rámcov.

Testovali sme proti dvom implementáciám echo serverov:
\begin{enumerate}
    \item Lokálny echo server založený na W3C príklade v Pythone \cite{w3c_echo_server}
    \item Rust wtransport knižnica \cite{wtransport_github}
\end{enumerate}

Detekcia pádov sa realizuje monitorovaním procesov a logov, identifikujúc anomálie ako paniky vlákien bez havárie celého servera.

Zdrojový kód nášho fuzzera je dostupný na adrese \url{https://github.com/qbitroot/webtransport-fuzzer}.
