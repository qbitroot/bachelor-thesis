\chapter{Súvisiacie práce a stav techniky}
\label{kap:suvisiace}

V tejto kapitole popisujeme súvisiace práce v oblasti fuzzingu protokolov a aktuálny stav techniky.

Oblasť protokolového fuzzingu sa významne vyvinula s rôznymi technikami zameranými na odhaľovanie zraniteľností v sieťových komunikáciách. Komplexný prehľad od Chen et al. \cite{chen2024survey} kategorizuje metódy protokolového fuzzingu do generačných, mutačných a hybridných prístupov, zdôrazňujúc úlohu stavovo-uvedomelého fuzzingu na manipuláciu komplexných interakcií protokolov. Autori diskutujú výzvy ako explózia stavov a potreba efektívnej voľby seedov, poznamenávajúc, že zatiaľ čo tradičné black-box fuzzeri ako Peach a Sulley boli základné, moderné nástroje zahŕňajú mechanizmy spätnej väzby inšpirované AFL na zlepšenie pokrytia. Zdôrazňujú, že protokolový fuzzing sa líši od všeobecného softvérového fuzzingu kvôli potrebe modelovania formátov správ a sekvencií, často vyžadujúc doménovo-špecifické znalosti na vyhnutie sa neplatným testovacím prípadom, ktoré sú okamžite odmietnuté. Prehľad tiež pokrýva metriky hodnotenia, ako sú miery objavovania chýb a hĺbka pokrytia, a poukazuje na obmedzenia v manipulácii šifrovaných protokolov ako QUIC, ktorý podlieha WebTransportu, kde sú potrebné dešifrovacie háky alebo proxying.

Nadväzujúc na to, štúdia od Wang et al. \cite{wang2024enhancing} sa zameriava na fuzzing pre sieťovú bezpečnosť, navrhujúc rámec, ktorý integruje strojové učenie pre inteligentnejšie stratégie mutácie. Rozlišujú black-box prístupy, ktoré sa spoliehajú iba na pozorovania vstup-výstup, od gray-box (s čiastočnou znalosťou) a white-box (plná inštrumentácia), argumentujúc, že gray-box dosahuje rovnováhu pre protokoly, kde je prístup k zdrojovému kódu obmedzený. V ich experimentoch testovali proti implementáciám HTTP/2 a TLS, zistiac, že duplikácia sekvencií a preusporiadanie rámcov odhalilo viac stavovo-súvisiacich chýb než náhodné mutácie. Poznamenávajú však medzery vo fuzzingu multiplexovaných protokolov, kde súbežné streamy komplikujú sledovanie stavov, čo je priamo relevantné pre dizajn WebTransportu. Práca kritizuje existujúce nástroje za chýbajúcu podporu pre vznikajúce IETF štandardy, navrhujúc, že sú často potrebné vlastné rozšírenia, ako bolo urobené v tejto práci s BooFuzz.

Daniele et al. \cite{daniele2024libafl} predstavujú LibAFL*, pokročilú fuzzingovú knižnicu, ktorá podporuje vlastné harnessy pre testovanie protokolov, demonštrujúc jej použitie na priemyselných riadiacich systémoch. Zdôrazňujú dôležitosť neinvazívneho monitorovania, ako je analýza logov, nad invazívnu inštrumentáciu, čo sa zhoduje s naším jazykovo-agnostickým prístupom. Práca porovnáva LibAFL* s BooFuzz, poznamenávajúc silné stránky BooFuzz v jednoduchosti použitia pre Python-based skriptovanie, ale slabiny v manipulácii vysokorýchlostných protokolov bez asynchrónnych integrácií. Ich zistenia ukazujú, že bez inštrumentácie sa detekcia pádov spolieha na externé signály ako ukončenie procesu alebo chyby v logoch, čo sme prijali pre servery WebTransportu. Kľúčová medzera identifikovaná je nedostatok fuzzerov pre webovo-orientované protokoly nad HTTP, bez zmienky o WebTransporte, čo podčiarkuje novinku nášho príspevku.

Napokon, recentný preprint od Li et al. \cite{li2024fuzzing} prehľadáva techniky fuzzingu pre IoT a sieťové protokoly, zdôrazňujúc vzostup model-based fuzzingu, kde špecifikácie protokolov (napr. z IETF draftov) vedú generovanie testovacích prípadov. Diskutujú nástroje ako Fuzzowski a BooFuzz, chváliac BooFuzz za jeho block-based modelovanie správ, ale kritizujúc jeho chýbajúcu vstavanú podporu pre mutácie sekvencií ako duplikácia, čo vyžadovalo modifikácie v našej implementácii. Autori poukazujú na to, že zatiaľ čo fuzzing bol extenzívne aplikovaný na zrelé protokoly ako TCP/IP alebo MQTT, vznikajúce ako WebTransport -- stále v draft štádiu, ale implementované v prehliadačoch -- zostávajú nedostatočne študované, s potenciálnymi zraniteľnosťami v integrácii QUIC neotestovanými. Navrhujú komunitne-riadené nástroje na GitHube, podobné nášmu projektu, na zaplnenie týchto medzier.

\section{Porovnanie s našou prácou}

V porovnaní s týmito prácami naša práca rieši špecifickú medzeru vo fuzzingu WebTransportu, ktorú žiadna z prehľadových prác priamo nerieši, pravdepodobne kvôli novosti protokolu. Zatiaľ čo prehľady ako \cite{chen2024survey} a \cite{li2024fuzzing} poskytujú široké prehľady, neobsahujú príklady WebTransportu, a nástroje ako BooFuzz vyžadujú rozšírenia pre protokoly založené na QUIC, ako sme implementovali použitím aioquic. Na rozdiel od white-box prístupov v \cite{daniele2024libafl} náš black/gray-box metóda vyhýba sa jazykovo-špecifickej inštrumentácii, robí ho univerzálnejším, ale potenciálne menej hlbokým v pokrytí; avšak sa osvedčil v objavovaní problémov ako dosiahnuteľná aserčná chyba v Rust wtransport knižnici \cite{wtransport_github}, kde odoslanie drain kapsule bez správneho HTTP/3 rámca spôsobilo paniku pracovného vlákna. Toto kontrastuje so súvisiacimi štúdiami fuzzingu na HTTP/3, ktoré sa zameriavajú na klientovú stranu, ale prehliadajú serverové echo implementácie.

Celkovo stav techniky odhaľuje zrelý ekosystém pre protokolový fuzzing, ale s obmedzenou aplikáciou na nové webové transporty, čo motivovalo náš vývoj BooFuzz-based fuzzeru prispôsobeného pre štruktúry správ a sekvencie WebTransportu.
