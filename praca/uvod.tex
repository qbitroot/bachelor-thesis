\chapter*{Úvod}
\addcontentsline{toc}{chapter}{Úvod}
\markboth{Úvod}{Úvod}

Fuzzing protokolov je dynamická testovacia technika používaná na objavovanie zraniteľností v softvérových implementáciách poskytovaním neplatných, neočakávaných alebo náhodných dát ako vstupov do programu, najmä v kontexte sieťových protokolov, kde pomáha odhaľovať chyby, ktoré statická analýza alebo tradičné testovanie môžu prehliadnuť. Fuzzing nie je náhradou formálnej verifikácie alebo vyčerpávajúceho testovania, ale ich dopĺňa simuláciou reálnych podmienok útočníka, často odhaľujúc okrajové prípady ako pretečenia bufferov, pády alebo logické chyby vznikajúce z deformovaných správ alebo neočakávaných sekvencií. Funguje na základe spätnej väzby, kde sa testovacie prípady generujú, vykonávajú a mutujú na základe pokrytia alebo detekcie pádov, čo ho robí obzvlášť efektívnym pre komplexné systémy ako sieťové stacky. Fuzzing však nie je neomylný; môže prehliadnuť jemné problémy založené na časovaní alebo vyžadovať značné výpočtové zdroje, a jeho efektivita závisí od kvality počiatočných seed vstupov a stratégií mutácie.

WebTransport je moderné webové API, ktoré umožňuje nízko-latentnú, bidirekcionálnu a multiplexovanú komunikáciu medzi webovými klientmi a servermi, stavané na HTTP/3 a transportnom protokole QUIC \cite{webtransport_http3, webtransport_api}, poskytujúce funkcie ako streamy, datagramy a relácie pre aplikácie ako hranie hier, živé streamovanie a reálnu spoluprácu. Na vysokej úrovni WebTransport nadväzuje reláciu cez QUIC, ktorý riadi kontrolu preťaženia, multiplexovanie a obnovu strát, zatiaľ čo HTTP/3 poskytuje rámcovanie pre počiatočné handshaky a kapsule; protokol riadi stavy vrátane nadviazania spojenia, vytvárania streamov, prenosu dát a elegantného ukončenia, s manipuláciou chýb pre neplatné rámce alebo kapsule. Hoci WebTransport je stále aktívne vyvíjaný podľa najnovších IETF draftov, je už široko podporovaný v hlavných prehliadačoch ako Chrome, Firefox a Safari, čo odráža jeho rýchle prijatie na zlepšenie webového výkonu nad tradičné WebSockets. Táto novinka znamená, že v tejto oblasti nie je veľa práce s fuzzingom, čo predstavuje príležitosť prispieť k jeho spoľahlivosti v ranom štádiu životného cyklu.

Zvolený black-box/gray-box fuzzingový prístup je výhodný, pretože považuje cieľ za čiernu skrinku bez potreby prístupu k zdrojovému kódu alebo inštrumentácie, čo ho robí jazykovo-agnostickým a vhodným pre testovanie rôznych implementácií ako tie v Pythone alebo Ruste, zatiaľ čo zahŕňa gray-box prvky cez znalosť špecifikácie protokolu na vedenie mutácií. Tento metóda je dobrá pre WebTransport, pretože umožňuje sústrediť sa na štruktúry správ a sekvencie bez hlbokej integrácie do špecifických runtimeov, umožňujúc širšiu aplikovateľnosť naprieč knižnicami serverov. Rozhodli sme sa vytvoriť univerzálny fuzzer namiesto jazykovo-špecifického s hlbokou inštrumentáciou, pretože to podporuje opätovné použitie, vyhýba sa závislostiam na interných stavoch, ktoré sa môžu líšiť podľa implementácie, a zodpovedá cieľu testovania echo serverov cez externé logovanie a detekciu pádov, zabezpečujúc, že fuzzer môže evoluovať s vývojom protokolu.

Ciele práce zahŕňajú vývoj fuzzeru pomocou BooFuzz \cite{boofuzz_docs, boofuzz_github} s mutáciou štruktúry správ a duplikáciou sekvencií, testovanie proti echo serverom a analýzu výsledkov na identifikáciu zraniteľností, riešiac výzvy ako manipulácia šifrovania QUIC a evolučnej povahy protokolu.

V kapitole~\ref{kap:suvisiace} popisujeme súvisiace práce a stav techniky v oblasti fuzzingu protokolov. V kapitole~\ref{kap:metodologia} predstavujeme prehľad protokolu WebTransport, metodológiu fuzzingu a našu implementáciu. V kapitole~\ref{kap:vysledky} prezentujeme výsledky, analýzu a diskusiu.
