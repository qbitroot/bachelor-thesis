\chapter{Výsledky, analýza a diskusia}
\label{kap:vysledky}

V tejto kapitole prezentujeme výsledky fuzzingových kampaní, ich analýzu a diskusiu o obmedzeniach a možných vylepšeniach.

\section{Výsledky fuzzingových kampaní}

Fuzzingové kampane sú v súčasnosti v prebiehajúcej fáze. Testujeme implementácie WebTransportu s cieľom identifikovať potenciálne chyby a zraniteľnosti. Výsledky budú systematicky zaznamenávané a kategorizované podľa typu zisteného problému.

\section{Analýza}

Analýza výsledkov sa zameriava na hodnotenie efektivity rôznych mutačných stratégií. Skúmame, ktoré typy mutácií sú najúčinnejšie pri odhaľovaní chýb v implementáciách WebTransportu. Porovnávame výsledky mutácií štruktúry správ s duplikáciou sekvencií a hodnotíme ich prínos pre testovanie protokolu.

\section{Diskusia}

\subsection{Obmedzenia}

Obmedzenia nášho prístupu zahŕňajú:
\begin{itemize}
    \item Závislosť na externých logoch namiesto internej inštrumentácie, čo môže prehliadnuť tiché chyby
    \item Výzvy s šifrovaním QUIC vyžadujúce špecializované háky
    \item Obmedzené pokrytie v porovnaní s white-box prístupmi
\end{itemize}

\subsection{Porovnanie so súvisiacimi prácami}

V porovnaní so súvisiacimi štúdiami fuzzingu ako v \cite{li2024fuzzing} náš prístup je univerzálnejší, ale menej optimalizovaný pre rýchlosť. Jazykovo-agnostický dizajn umožňuje testovanie rôznych implementácií bez potreby modifikácie fuzzera.

\subsection{Návrhy na vylepšenia}

Navrhujeme nasledujúce vylepšenia pre budúcu prácu:
\begin{itemize}
    \item Rozšírenie na ďalšie implementácie WebTransportu
    \item Fuzzing klientskej strany v prehliadačoch
\end{itemize}
