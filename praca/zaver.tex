\chapter*{Záver}
\addcontentsline{toc}{chapter}{Záver}
\markboth{Záver}{Záver}

Táto práca sa zaoberala fuzzingom protokolu WebTransport s cieľom identifikovať chyby v implementáciách a zraniteľnosti. Predstavili sme vyvinutý fuzzer založený na BooFuzz, ktorý zahŕňa mutácie štruktúry správ a duplikáciu sekvencií.

Kľúčové dosiahnutia tejto práce zahŕňajú:
\begin{itemize}
    \item Vývoj univerzálneho fuzzera pre protokol WebTransport s integráciou aioquic pre podporu QUIC
    \item Aplikáciu techník mutácie a duplikácie sekvencií prispôsobených pre WebTransport
    \item Analýzu výsledkov odhaľujúcich chyby v implementáciách, vrátane dosiahnuteľnej aserčnej chyby v Rust wtransport knižnici
    \item Demonštráciu efektivity black/gray-box prístupu v odhaľovaní paník v pracovných vláknach
\end{itemize}

Náš príspevok posilňuje testovanie protokolov a prispieva k spoľahlivosti WebTransportu v ranom štádiu jeho životného cyklu. Práca odráža dôležité lekcie o význame jazykovo-agnostického prístupu pri testovaní evolučných štandardov.

Širšie dopady na sieťovú bezpečnosť podčiarkujú potrebu skorého fuzzingu v nových webových technológiách na prevenciu zraniteľností v široko prijímaných systémoch. Keďže WebTransport je už implementovaný v hlavných prehliadačoch, ale stále sa aktívne vyvíja, včasná identifikácia a oprava chýb je kritická pre jeho bezpečné nasadenie.

Budúca práca by mohla zahŕňať rozšírenie fuzzera o inteligentnejšie stratégie mutácie využívajúce strojové učenie, testovanie klientskej strany v prehliadačoch a aplikáciu na ďalšie implementácie protokolu. Zdrojový kód fuzzera je dostupný na stránke \url{https://github.com/qbitroot/webtransport-fuzzer}.
